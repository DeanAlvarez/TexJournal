\documentclass{tufte-handout}
% tufte-latex handles page geometry (wide margins for sidenotes) and fonts.
% Default fonts are typically Palatino for text, Bera Mono for code, Euler for math.

% Tufte-latex loads many common packages like geometry, graphicx, xcolor, amsmath, hyperref by default.
% You generally don't need to load them again unless you need specific options not set by tufte-latex.

% If you wish to use Latin Modern fonts instead of Tufte's defaults:
% \usepackage{lmodern}
% \usepackage[T1]{fontenc} % Recommended with lmodern

% \usepackage[utf8]{inputenc} % Usually handled by modern TeX engines or tufte-latex itself

% --- Document Information ---
\title{Journal Entry}      % You can make this more dynamic later if needed
\author{}                  % Typically not needed for a personal journal, so leave blank or omit
\date{%%DATE%%}            % Placeholder for the date

% --- Hyperref Customization (tufte-latex loads hyperref) ---
% You can customize link colors if desired. Tufte style is often subtle.
% \hypersetup{
%   colorlinks=true,
%   linkcolor=Maroon, % Example of a more Tufte-esque color
%   urlcolor=NavyBlue,
%   citecolor=OliveGreen
% }

% \setlength{\parindent}{1em} % tufte-latex often has no paragraph indentation by default. Add if desired.
% \usepackage{setspace} % If you need to adjust line spacing, tufte-latex has its own defaults.
% \onehalfspacing

\begin{document}

\maketitle % This will display the title, author (if any), and date.

% After \maketitle, you might want a slight pause or a rule if the title feels too close.
% For example:
% \begin{fullwidth} % Use fullwidth environment if you want something to span the entire page width
% \hrulefill
% \end{fullwidth}
% \vspace{1em}


% --- Content Placeholder ---
% Your journal content (pure LaTeX) goes here.
%
% Using Tufte-style features:
% - For margin notes with a number in the text: \sidenote{Your note here.}
% - For unnumbered notes directly in the margin: \marginnote{Your note here.}
% - To start a paragraph with emphasized first few words: \newthought{The first words of your new thought...}
% - For full-width figures: \begin{figure*} ... \end{figure*}
% - For margin figures: \begin{marginfigure} ... \end{marginfigure}
%
%%CONTENT%%


\end{document}